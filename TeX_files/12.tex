\section{Поляризация отраженной и преломленной волн. Степень поляризации отраженного и преломленного света. Угол Брюстера. Физический смысл закона Брюстера.}

\textbf{Поляризация отражённой и преломлённой волн.}

	\begin{figure}[H]
	\centering
	\includegraphics*[width=\textwidth]{Otr}
    \end{figure}

$\phi$ --- угол падения, $\psi$ --- угол преломления;
$E_i$ --- амплитуда падающей волны, $E_r$ --- прошедшей.

Степень поляризации $$\boxed{\Delta = \frac{I_\perp - I_\parallel}{I_\perp + I_\parallel}}$$

Отношение отражённого потока к падающему:
$$r^2_\perp = [\frac{\sin{(\phi - \psi)}}{\sin{(\phi + \psi)}}]^2, \ r^2_\parallel = [\frac{\tan{(\phi - \psi)}}{\tan{(\phi + \psi)}}]^2$$

При $\phi \to \pi/2$ (скользящее падения): $r^2_\perp = r^2_\parallel = 1$ --- полное отражение.

При полном отражении фаза волны испытывает скачок $\delta_\perp$ и $\delta_\parallel$:
$$\tan{\frac{\delta_\perp}{2}} = \frac{\sqrt{\sin^2{\phi}-n^2}}{{\cos{\phi}}}, \ \tan{\frac{\delta_\parallel}{2}} = \frac{\sqrt{\sin^2{\phi}-n^2}}{{n^2\cos{\phi}}},$$
		
		Компоненты $E_r\perp$ и $E_r\parallel$ испытывают изменения фазы по отношению к $E_i\perp$ и $E_i\parallel$, причем $\delta_\perp$ отлично от $\delta_\parallel$. Для их разности справедливо $$\tan{\frac \delta 2 = \frac {\cos{\phi}\sqrt{sin^2{\phi} - n^2}}{\sin^2{\phi}}} \Rightarrow \tan{\frac \delta 2} > 0, \ 0< \delta< \pi.$$
		
		Если в падающей волне $E_i\perp$ и $E_i\parallel$ находятся в одной фазе, то в отраженном свете между $E_r\perp$ и $E_r\parallel$ появляется сдвиг фазы, зависящий от угла падения и показателя преломления. Таким образом, \textit{явление полного внутреннего отражения позволяет получить эллиптически-поляризованный свет.} Учитывая $0< \delta< \pi$, эллиптическая поляризация будет \textit{левой}.
		
		$\delta = 0$ при $\phi = \phi_{critical}$ и $\phi = \pi/2$ 
		Максимальный сдвиг $\delta_m$ при $\cos\phi = \sqrt \frac {(1 - n^2)}{(1+n^2)}$: $\tan \frac {\delta_m} 2 = \frac{1-n^2}{2n}$
		
		Для получения \textit{круговой поляризации} отражённого света, необходимо выполнение условий: $$E_r\parallel = \pm E_r\perp, \ \delta = \pi/2.$$
		Чтобы получить такой сдвиг при однократном отражении нужен $n = \sqrt2 - 1.$
		Для стекла можно подобрать такие значения угла падения, чтобы $\delta = \pi/4$, тогда при двукратном полном отражении в стекле происходило изменение фазы на $\pi/2$ (так действует пластинка в четверть волны).
		
	\begin{figure}[H]
	\centering
	\includegraphics*[width=0.4\textwidth]{ParFar}
    \end{figure}
Если $E_i\parallel = E_i\perp$ (плоскоть поляризаии составляет угол 45 градусов), то при полном внутреннем отражении $|E_r\parallel| = |E_r\perp|$ и при $\delta = \pi/2$ свет получится поляризованным по кругу.

\textbf{Явление Брюстера.}

Согласно \textit{формулам Френеля} коэффициент отражения параллельно поляризованной волны обращается в ноль при таком угле $\theta_B$, что $\tan{\theta_B = \frac{n_2}{n_1}}$, где $n_1$ --- показатель преломления среды, из которой падает луч, $n_2$ --- показатель преломления, в которую падает луч. Угол $\theta_B$ называется \textbf{углом Брюстера}.

	При падении под таким углом отражённая волна оказывается полностью поляризованной, обладающей перпендикулярной поляризацией.
	
	Коэффициент отражения для $p$-поляризованной волны (из формул Френеля):
	$$R_{||} = \frac{\tan^2{(\alpha - \beta)}}{\tan^2{(\alpha + \beta)}}, $$
	где  $\alpha$ --- угол падения, $\beta$ --- угол преломления.
	
	Из равенства нулю $R_{||}$ следует, что $\theta_B + \beta = \frac{\pi}{2}$.
	
	\textbf{Физический смысл закона Брюстера}
	
	Отражённая волна возникает вследствие того, что излучение, падающее не вещество, проникает в него и возбуждает колебания электронов. Возникающие в результате этого волны от всех электронов суммируются и формируют отражённую волну. Однако при падении луча под углом Брюстера преломлённый и отражённый лучи образуют прямой угол. При параллельной поляризации волны вектор напряжённости поля преломлённой волны $E''$ совершает колебания в направлении, параллельном волновому вектору отражённой волны. Однако, как видно из диаграммы направленности (на рисунке ниже --- справа), именно в этом направлении интенсивность излучения практически нулевая, что приводит к отсутствию отражённого излучения.
	\begin{figure}[H]
		\centering
		\includegraphics*[width=\textwidth]{Bruster}
	\end{figure}

\textbf{Получение линейно поляризованного света.}

При падении света на одну пластинку под углом Брюстера интенсивность отражённого линейно поляризованного света очень мала (от границы воздух--стекло отражается около 3,75\% интенсивности падающего луча). Для прошедшего излучения отношения интенсивностей с параллельной и перпендикулярной поляризациями равно $$\delta = \frac{(I_d)_\perp}{(I_d)_\parallel} = \frac{d^2_\perp}{d^2_\parallel},$$
где $d_\perp$ и $d_\parallel$ --- амплитудные коэффициенты прохождения.

$$d_\perp = \frac{2\sin{\alpha}\cos{\beta}}{\sin{(\alpha + \beta)}}, \ d_\parallel = \frac{2\sin{\alpha}\cos{\beta}}{\sin{(\alpha+\beta)}\cos{(\alpha - \beta)}}$$

Следовательно $\delta = \cos^2{(\alpha - \beta)},$ при $\alpha = \theta_B$: $$\delta = \frac{4n^2}{(1+n^2)^2} < 1.$$ 
Таким образом, в прошедшем свете доля перпендикулярной компоненты уменьшается. Для увеличения степени поляризации прошедшего и отражённого света применяют несколько пластинок, сложенных в \textit{стопу Столетова}. При прохождении её, отражённый свет становится перпендикулярно поляризованным, а прошедший - параллельно поляризованным. Так, для стопы с 16 пластинками с $n = 1,5$ степеь поляризации около 99\%.

	\begin{figure}[H]
	\centering
	\includegraphics*[width=0.4\textwidth]{Stoletov}
    \end{figure}